%! TEX root = ../main.tex
\documentclass[main]{subfiles}

\begin{document}
\chapter{結言}
\label{ch:conclusion}

\section{本研究のまとめ}
本研究では,農業用ハウス内を巡回して計測を行う自動走行システムを開発し,
GNSSの測位精度が不安定な環境における自己位置推定の安定化と,
高速分光センシングを用いた環境情報の可視化について検討した.

まず,ハウス環境におけるRTK-GNSSの挙動を調査した.
その結果,多くの時間帯では高精度な測位ができるものの,
遮蔽物やマルチパスの影響により,受信機ステータスが「RTK-FIX」等の良好な状態を示していても,
数メートル規模の突発的な誤差が発生し得ることを実測データから確認した.
このような突発的な外れ値が地図作成に入力されると,
自己位置推定および地図の幾何学的整合性が破綻することが課題であった.

次に,この課題への解決策として,LIO(LiDAR Inertial Odometry)を
短時間の基準としたGNSS品質監視手法を構築した.
本手法は,GNSSの観測値をそのまま採用するのではなく,
LIOの増分変位とGNSSの増分変位の動的整合性を評価することで,
観測値の採択または遮断を判定するアーキテクチャである.
実環境の走行データを用いた検証により,NLOSの影響が疑われる区間において適切にGNSSデータを遮断することで,
地図の大きな歪みを防げることを確認した.
ただし,今回の検証は限られたデータに基づくものであり,
手法の汎用性や定量的な性能評価については今後の課題である.

また,より高密度な環境計測を実現するため,
小型分光センサC12880MAの高速駆動方式を開発した.
従来の割込み処理による制御では高速化に限界があることを示した上で,
ハードウェアトリガとDMAを用いた取得方式を実装した.
これにより,STM32H723ZGにおいて\SI{5}{\mega\hertz}での安定した連続取得を実現した.
さらに,データの読み出し開始時に生じる画素インデックスの固定オフセットを補正することで,
高速取得時でも正確なスペクトルが得られるようにした.

加えて,上記の自己位置推定システムとセンシングシステムを統合し,
温度・湿度・CO$_2$ 濃度・気圧,および分光データの特徴量を,
共通の空間座標系上に重畳する処理系を構築した.
これにより,巡回計測から得られる非同期な環境データを幾何学的に正しい軌跡上へマッピングし,
圃場内の微気象および植生状態の相対的な差異を空間分布として可視化できることを示した.

なお,本研究では自律走行ための3D-LiDARを用いたナビゲーションシステムを導入し,
研究室内での基本的な動作確認までは行った.
しかし,実際の圃場環境における自律走行実験については,
時間的な制約により十分な検証ができておらず,
本論文では自動走行システムの構築と要素技術の検証に留まっている.

結言として,本研究はGNSS劣化環境下における移動ロボットの
自己位置推定と環境センシングの統合アーキテクチャを確立した.
特に,受信機内部の信頼度指標では判別不可能なNLOS起因の大規模誤差を
LIOとの動的整合性により棄却する品質監視モジュールは,
地図の不整合を防ぎ,農業用マッピングシステムを実現するための有効なアプローチであると結論づける.

\section{今後の展望}

今後の展望として,まず,
検出精度の定量的な評価である.
本研究では正確な基準軌跡との比較が行えていないため,
今後は測量機器などの信頼できるデータを用いて,
提案手法の検知感度および誤検知率を統計的かつ定量的に評価する

また,判定ロジックの拡張である.
本研究ではGNSSの異常のみを判定対象としたが,
LiDARスキャンマッチングの幾何学的縮退する場所や,IMUの誤差といった他の要因も考慮する必要がある.
複数のセンサの異常を総合的に判断することで,
実環境における信頼性をさらに向上させることが期待される.

さらに,自動走行システムと統合した検証である.
実際にロボットを自動走行させながら長時間の実験を行い,
環境の変化に対してシステムが安定して動作するかを確認する必要がある.
これにより,実用化に向けて解決すべき課題がより明確になると考えられる.


\end{document}