%! TEX root = ../main.tex
\documentclass[main]{subfiles}

\begin{document}
\chapter{結言}
\label{ch:conclusion}

\section{本研究のまとめ}
本研究では,農業用ハウス内を巡回して計測を行う自動走行システムを開発し,
GNSSの測位精度が不安定な環境における自己位置推定の安定化と,
高速分光センシングを用いた環境情報の可視化について検討した.

まず,ハウス環境におけるRTK-GNSSの挙動を調査した.
その結果,多くの時間帯では高精度な測位ができるものの,
遮蔽物やマルチパスの影響により,受信機の状態が良好であっても大きな誤差が
発生する場合があることを実測データから確認した.
このような突発的な外れ値が地図作成に入力されると,
地図の整合性が大きく損なわれることが課題であった.

次に,この課題への解決策として,LIO(LiDAR Inertial Odometry)を
短時間の基準としたGNSS品質監視手法を構築した.
本手法は,GNSSの観測値をそのまま採用するのではなく,
移動量の整合性をチェックして採否を判定する仕組みである.
実環境の走行データを用いた検証により,NLOSの影響が疑われる区間において適切にGNSSデータを遮断することで,
地図の大きな歪みを防げることを確認した.
ただし,今回の検証は限られたデータに基づくものであり,
手法の汎用性や定量的な性能評価については今後の課題である.

また,より高密度な環境計測を実現するため,
小型分光センサC12880MAの高速駆動方式を開発した.
従来の割込み処理による制御では高速化に限界があることを示した上で,
ハードウェアトリガとDMAを用いた取得方式を実装した.
これにより,STM32H723ZGにおいて\SI{5}{\mega\hertz}での安定した連続取得を実現した.
さらに,データの読み出し開始時に生じる画素位置のズレを補正することで,
高速取得時でも正確なスペクトルが得られるようにした.

加えて,上記の自己位置推定システムとセンシングシステムを統合し,
温度・湿度・CO$_2$ 濃度・気圧,および分光データの特徴量を,
空間上の位置と対応付けてマッピングする処理系を構築した.
これにより,巡回計測によって得られた多様な環境情報を,
位置情報と紐付けて可視化できることを示した.

なお,本研究では自律走行ための3D-LiDARを用いたナビゲーションシステムを導入し,
研究室内での基本的な動作確認までは行った.
しかし,実際の圃場環境における自律走行実験については,
時間的な制約により十分な検証ができておらず,
本論文では自動走行システムの構築と要素技術の検証に留まっている.

以上より,本研究は,GNSSが劣化しやすい農業環境における
自己位置推定と環境センシングの統合システムについて,
基礎的な設計・実装を行い,実データに基づく検証を行ったものである.
特に,見かけ上は良好なGNSS観測に含まれる外れ値に対し,
実際には大きくズレているデータを検出し,
地図作成への悪影響を防ぐ方法を提案した.


\section{今後の課題}

今後の課題として,まず,
検出精度の定量的な評価である.
本研究では正確な基準軌跡との比較が行えていないため,
今後は測量機器などの信頼できるデータを用いて,
提案手法がどれくらい正確に異常を検知できるかを数値で評価する必要がある.

また,判定ロジックの拡張である.
本研究ではGNSSの異常のみを判定対象としたが,
LiDARの特徴量が不足する場所や,IMUの誤差といった他の要因も考慮する必要がある.
複数のセンサの異常を総合的に判断することで,
実環境における信頼性をさらに向上させることが期待される.

さらに,自動走行システムと統合した検証である.
実際にロボットを自動走行させながら長時間の実験を行い,
環境の変化に対してシステムが安定して動作するかを確認する必要がある.
これにより,実用化に向けて解決すべき課題がより明確になると考えられる.


\end{document}