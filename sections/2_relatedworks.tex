%! TEX root = ../main.tex
\documentclass[main]{subfiles}
\begin{document}
\chapter{関連研究}
本研究の目的は,農業ロボットの自動走行において,
信頼性の高い自己位置推定と環境センシングを統合し,
生育環境マップを生成することである.
本章では,巡回システム,自己位置推定,GNSSの品質監視,
および農業センシングに関する先行研究を整理し,本研究の立ち位置を明確にする.

\section{圃場巡回システム}
本研究室の先行研究では,AGVを用いた桑畑巡回システムが提案されている\cite{ref:Kobayashi2021AGV}.
同研究は,圃場での巡回観測の有効性を示した一方で,自律走行は主にシミュレーション上で検証され,
実機における安定運用の確立が課題として残存していた.
また,位置情報としてGNSSを利用していたものの,RTKによる補正は導入されていなかった.

これに対し著者らは,2D-LiDAR,IMU,およびホイールオドメトリを
用いたAGVの自動走行を実現した\cite{ref:xu2024development}.
しかし,2D-LiDARによるSLAMは,ハウスのような環境
では自己位置推定が不安定化する傾向にある.また,
2D計測では高さ情報を取得できないため,
三次元的な障害物の正確な認識が困難である.
さらに,車輪の空転やスリップによって誤差が蓄積するという課題もあった.
そこで本研究では,3D-LiDARとIMUによる自己位置推定を主軸とし,
RTK-GNSSによる位置補正を組み合わせる手法を採用する.
これにより,ハウス環境での巡回計測に必要な信頼性と安全性を確保する.


\section{農業環境における自己位置推定とGNSS誤差対策}

\subsection{LiDARを用いた自己位置推定と課題}
GNSS信号が不安定な環境では,3D-LiDARとIMUを組み合わせた
LIO(LiDAR Inertial Odometry)が自己位置推定の中核となる.
LOAM \cite{ref:loam}以降,LIO-SAM \cite{ref:lio_sam} などの最適化手法が発展した.
近年では計算効率に優れたFAST-LIO2\cite{ref:fast_lio2} が広く利用されている.

しかし,農業用ハウスのような環境では,
直線的な通路が続くため縮退と呼ばれる現象が発生しやすい.
これは,進行方向における空間的特徴の欠如により,スキャンマッチングの拘束条件が不足する現象である.
その結果,長時間走行でドリフトが蓄積し,地図が歪む原因となる\cite{10.3389/frobt.2022.832165}.
したがって,LIO単独の推定に頼らず,外部基準を用いてドリフトを抑制する手法が重要となる.

\subsection{GNSS誤差に対する既存のアプローチ}

GNSSを用いて絶対位置を提供する場合,
ハウス環境特有のNLOSやマルチパスが大きな問題となる.
本研究では,平均的な位置精度だけでなく,
外れ値による軌跡の不連続や,地図の不整合が回避する推定のロバスト性を重視する.
また,位置の誤りが一度地図や環境マップに書き込まれると,
後から補正しても誤配置が残り得るため,
地図の不整合の確実な防止が求められる.
この観点から,既存研究の対策は処理の段階により大きく3つに分類できる.

第一のアプローチは観測段階での選別である.
環境の事前情報を使って,測位に使う衛星を選ぶ方法である.
都市部では,3D建物モデルを用いて遮蔽される衛星を予測する
Shadow Matchingが提案されている\cite{Groves2011ShadowMatching}.
また,車載LiDARで周囲の形状を計測し,NLOSの影響を下げる研究もある\cite{Wen2022LidarAidedNLOS}.
しかし,これらの手法は事前に正確な地図があることや,建物のような明確な幾何学的特徴があることを前提としている.
農業用ハウスや圃場では,事前に精密な3D地図を用意することは難しく,作物の成長によって環境も変化する.
そのため,これらの手法をそのまま適用することは困難である.

第二のアプローチは推定段階でのロバスト化である.
自己位置推定の計算の中で,外れ値の影響を弱める方法である.
因子グラフ最適化において,誤差の大きい拘束の重みを下げる
Switchable ConstraintsやDCS(Dynamic Covariance Scaling)が提案されている
\cite{Suenderhauf2012Switchable,Agarwal2013DCS}.
これらは,外れ値が散発的に起きる場合に有効である.
一方で,ハウス内では NLOS による誤差が一定時間続くことがある.
このような誤差が連続して入力されると,
最適化がそれに引きずられ,
重み係数が十分に減衰する前に軌跡および地図構造が破綻する懸念がある.
したがって,推定段階だけで地図の不整合を確実に防ぐことは難しい.

第三のアプローチはGNSSの品質監視である.
推定結果の信頼度を監視し,危険な更新を止める方法である.
航空分野では RAIM/ARAIM が整理されており,
冗長な衛星観測に基づいて異常を検知し,信頼性を評価する
\cite{Blanch2015ARAIM}.
しかし,これは上空が開けている状況を前提とすることが多い.
地上の遮蔽環境では,可視衛星数が減り,
複数の衛星が同時に影響を受ける場合もあるため,
同じ仮定が成り立ちにくい.
また,農業分野でのマルチセンサ統合も報告されているが
\cite{Yan2022Greenhouse},
多くはカルマンフィルタによる単純な統合であり,
外れ値により地図や環境マップが不整合されるリスクを
主要な評価対象として扱っていない例が多い.

以上を踏まえると,
農業用ハウスのような GNSS 劣悪環境では,
事前地図を前提としないこと,
および外れ値を地図生成に入れる前に止めることが重要となる.
次節では,比較対象として
GNSS 単体,品質監視なしの単純融合,および提案手法を置き,
安定性と地図の不整合の観点で差が出る点を整理する.

\subsection{比較による位置づけ}
本研究では,自己位置推定の安定性を,
外れ値による軌跡の不連続が生じにくいこと,
および地図・環境マップの影響が生じにくいこととして扱う.
本研究の立ち位置を明確にするため,比較対象としてGNSS単体,
品質監視を行わない単純融合,
ならびに代表的なロバスト化手法を置き,
処理段階と地図の不整合リスクの観点から整理した結果をTable~\ref{tab:related_comparison_stability} に示す.

\begin{table}[t]
  \centering
  \caption{Comparison of GNSS Outlier Mitigation Methods and the Positioning of This Study}
  \label{tab:related_comparison_stability}
  \small
  \begin{tabular}{p{2.8cm} p{2.4cm} p{2.6cm} p{2.2cm} p{3.0cm}}
    \hline
    手法 & 処理段階 & 外れ値への対応 & 安定性\\
    \hline \hline
    GNSS単体 &
    観測・測位 &
    受信機出力をそのまま使用 &
    不連続が発生しやすい\\
    \hline
    単純な統合&
    推定 &
    誤差が正規分布と仮定 &
    外れ値に弱い\\
    \hline
    観測段階での選別&
    観測 &
    事前地図等でNLOSを排除 &
    環境変化に弱い(事前地図が必須\cite{Groves2011ShadowMatching}.)\\
    \hline
    グラフ最適化&
    最適化&
    拘束の重みを事後的に低減&
    連続外れ値で崩れる可能性あり\cite{Suenderhauf2012Switchable,Agarwal2013DCS}\\
    \hline
    本研究 \par (LIO基準の品質監視) &
    前処理 \par (入力遮断) &
    LIOとの整合で棄却 &
    不整合を抑制\\
    \hline
  \end{tabular}
\end{table}

Table~\ref{tab:related_comparison_stability} より,
品質監視を行わない統合では,観測外れ値が推定結果および地図へ直接的に悪影響を及ぼす.
また,ロバスト最適化は有効な手段であるが,計算コストが高く,誤ったデータが一度グラフに入ってしまうリスクがある.
一方で,本研究は入力段階で不正な GNSS 更新を遮断するため,
地図の不整合となる拘束の混入を未然に防げる点に特徴がある.



\section{農業用生育環境センシングセンサと分光計測}
環境情報を地図上に正しく配置するためには,
計測時の自己位置が正確である必要がある.
自己位置の誤差はマップの品質に直結するため,
信頼性の高い自己位置推定は,高精度な生育環境マップを作成する上で不可欠な要素である.

精密農業において,圃場内の微気象(温度,湿度,CO$_2$濃度など)を把握することは重要である.
固定式のセンサノードは広く普及しているが\cite{ref:mobile_sensing_review},設置コストや電源確保の面から,
細かな空間分解能でデータを取得することには限界がある.
これに対し,移動ロボットにセンサを搭載して巡回計測を行うことで,
より高密度な環境マップを作成する試みが進んでいる.

また,作物の状態を傷つけずに診断する手法として,
分光反射率の計測が注目されている.近年では,MEMS技術を活用した小型分光センサ(C12880MA等)が登場し,
ドローンや移動ロボットへの実装が容易となっている\cite{ref:spectral}.

先行研究\cite{ref:Kobayashi2021AGV}では,このセンサを用いた計測システムが構築されたが,
移動しながらの連続的な計測が難しいことや,回路設計の制約によりデータ取得速度が\SI{50}{\kilo\hertz}程度に制限される課題が存在した.
しかし,センサ自体は\SI{}{\mega\hertz}オーダの高速動作が可能な設計である.
したがって,これらの制限はセンサ自体の性能ではなく,読み出し回路側の設計に起因するものと考えられる.



\section{本研究の位置づけと新規性}
農業用ハウス向け移動ロボットでは,固定レールやガイド追従により走行経路を拘束した方式がしばしば用いられる.
また,ハウス内のようにGNSSが利用しにくい環境では,UWB等の屋内測位やSLAMにより自己位置推定を行う研究が報告されている.
一方,ハウス周辺から露地を含む運用では,作業線の再現性や絶対座標付与の観点からRTK-GNSSを前提とする実用システムが存在する.
しかし遮蔽・反射の影響により,RTKであっても測位外れ値が生じ,自己位置推定が不安定化し得る.


本研究の新規性は,農業環境におけるGNSS品質の解析,
監視モジュールの提案,およびセンサユニットの統合にある.

まず,ハウス環境のデータからNLOS等の外れ値の発生傾向を整理し,
自己位置推定の破綻を招く条件を明確にした.これに基づき,
受信機の状態量だけでなくLIOとの整合性によってGNSS信号を評価する監視モジュールを提案し,
信号劣化時でも地図の破壊を防ぎつつ正確な位置補正を行う手法を実現した.
さらに,分光センサC12880MAの駆動方式を最適化することで \SI{5}{\mega\hertz} の高速サンプリングを達成し,
これらを統合して環境マッピングを実証する.

\section{用語の定義}
本節では,本文中で頻出する専門用語について,
本論文における意味を簡潔に定義する.


  \noindent\textbf{GNSS}:Global Navigation Satellite Systemの総称であり,本論文では,衛星測位に基づく位置観測を一般に指す場合にGNSSと表記する.

  \noindent\textbf{RTK-GNSS}:RTK(Real-Time Kinematic)差分補正を用いるGNSS測位を指す.本研究の受信機はRTK対応であるが,測位解は環境・補正情報の状態によりRTK-FIX / RTK-FLOAT / No-Fixへ遷移し得る.以降,RTK補正を前提とする受信機出力を述べる箇所ではRTK-GNSSと表記し,補正状態を区別しない一般論・総称として述べる箇所ではGNSSと表記する.

  \noindent\textbf{ROS 2}:Robot Operating System 2の略称であり,ロボットソフトウェア開発のためのミドルウェアである.

  \noindent\textbf{topic/トピック}:ROS 2における通信の単位であり,ROS 2ノード間でメッセージを送受信するための論理チャネルである.

  \noindent\textbf{ROS 2 bag(rosbag2)}:ROS 2におけるデータ記録機構であり,topic通信を時系列ログとして保存したデータを指す.本論文では,実験時に取得したセンサデータおよび推定結果の記録に用いる.

  \noindent\textbf{tf}:ROS 2において座標系間の位置・姿勢関係を管理する仕組みである.本研究では,\texttt{map},\texttt{odom},\texttt{base\_link}などの座標系を用いてロボットの状態を表現する.

  \noindent\textbf{cmd\_vel}:ロボットの速度指令を表すROS 2 topicであり,並進速度および角速度を含む.本研究では,自律走行モジュールから車体へ送信される制御入力として用いる.

  \noindent\textbf{SLAM}:Simultaneous Localization and Mappingの略称であり,自己位置推定と環境地図生成を同時に行う枠組みである.

  \noindent\textbf{Foxglove}:ROS 2の可視化ツールであり,実行中のノードやトピックをリアルタイムで監視・可視化する.本研究では,実験データの可視化およびデバッグに用いる.

  \noindent\textbf{NIR}:Near-Infraredの略称であり,分光センサで取得される近赤外線波長帯(およそ700~1400\,nm)を指す.

  \noindent\textbf{RED帯域}:Red波長帯(およそ600~700\,nm)を指す.

  \noindent\textbf{正規化差分(ND)}:正規化差分(Normalized Difference, ND)は,2つのスペクトル成分の相対的な差を表す指標であり,照度変化の影響を低減する目的で用いられる.本研究では,NIR帯域とRed帯域の値を用いて正規化差分を算出し,スペクトル形状の違いを相対的に評価する.


\end{document}