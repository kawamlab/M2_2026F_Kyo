%! TEX root = ../main.tex
\documentclass[main]{subfiles}
\begin{document}
\chapter{関連研究}
本研究の目的は,農業ロボットの自動走行において,
信頼性の高い自己位置推定と環境センシングを統合し,
生育環境マップを生成することである.
本章では,巡回システム,自己位置推定,GNSSの品質監視,
および農業センシングに関する先行研究を整理し,本研究の立ち位置を明確にする.

\section{圃場巡回システム}
本研究室の先行研究では,AGVを用いた桑畑巡回システムが提案されている\cite{ref:Kobayashi2021AGV}.
同研究は,圃場での巡回観測の有効性を示した一方で,自律走行は主にシミュレーション上で検証され,
実機における安定運用の観点では課題が残っていた.
また,位置情報としてGNSSを利用していたものの,RTKによる補正は導入されていなかった.

これに対し筆者は,2D-LiDAR,IMU,およびホイールオドメトリを
用いたAGVの自動走行を実現した\cite{ref:xu2024development}.
しかし,2D-LiDARによるSLAMは,ハウスのような環境
では自己位置推定が不安定になりやすい.また,
2D計測では高さ情報を取得できないため,
枝や棚などの障害物を正しく認識することが困難である.
さらに,車輪の空転やスリップによって誤差が蓄積するという課題もあった.
そこで本研究では,3D-LiDARとIMUによる自己位置推定を主軸とし,
RTK-GNSSによる位置補正を組み合わせる手法を採用する.
これにより,ハウス環境での巡回計測に必要な信頼性と安全性を確保する.

\section{農業環境における自己位置推定}
\subsection{LiDARを用いた自己位置推定}
GNSS信号が不安定な環境では,3D-LiDARとIMUを組み合わせた
LIOが自己位置推定の中核となる.
LOAM \cite{ref:loam} 以降,
LIO-SAM \cite{ref:lio_sam} などの最適化手法が発展し,
近年では計算効率に優れたFAST-LIO2 \cite{ref:fast_lio2} が広く利用されている.

しかし,農業用ハウスのような環境では,
直線的な通路が続くため縮退と呼ばれる現象が発生しやすい.
これは進行方向の特徴が不足することでスキャンマッチングが不安定になる現象であり,
長時間走行でドリフトが蓄積し,地図が歪む原因となる \cite{ref:greenhouse_lio}.
したがって,LIO単独の推定に頼らず,外部基準を用いてドリフトを抑制する手法が重要となる.

\subsection{GNSS/INS/LiDARのセンサフュージョンと課題}
LIOのドリフトを抑制するため,
RTK-GNSSによる絶対位置情報を統合する手法が一般的である.
多くの場合,拡張カルマンフィルタやグラフ最適化を用いてGNSSデータを統合するが,
これらの手法はGNSSの誤差が正規分布に従うことや受信機の状態が信頼できる
ことを前提としている\cite{ref:gnss_fusion_review}.

しかし,ハウス環境では金属フレーム等による電波の反射や遮蔽が発生する.
この影響で,受信機の表示が良好であっても,実際には数メートルの誤差が含まれることがある.
このような誤ったデータを最適化計算に導入すると,推定位置が大きく飛び,
地図全体が破壊される恐れがある.そのため,GNSSデータを動的に選別し,
安全に統合するための品質監視の枠組みが必要である.

\subsection{NLOS/マルチパスと完全性監視}
GNSSの誤差の主な要因として,
マルチパスとNLOS受信が挙げられる.マルチパスは直達波と反射波の干渉によるものだが,
NLOSは反射波のみを受信するため,測距誤差が大きなバイアスとなりやすい.
このため,受信機が出力する状態量のみでは外れ値を十分に判別できない場合がある.
先行研究\cite{ref:Kobayashi2021AGV}でも,
農業施設周辺でGNSS軌跡が局所的に乱れる事例が報告されている(Fig.~\ref{fig:kobayashi_multipath_example}).

\begin{figure}[t]
  \centering
  \includegraphics[width=0.85\linewidth]{figures/2/kobayashi_map.pdf}
  \caption{GNSS trajectory interference in farming. Segments 1 and 2 are near 
  farmland and facilities, where environment degrades accuracy.\cite{ref:Kobayashi2021AGV}}
  \label{fig:kobayashi_multipath_example}
\end{figure}


完全性監視は,測位精度が要求を満たさない場合に警告や排除を行う仕組みである.
航空分野ではRAIM等が普及しているが,地上環境では障害物の影響が大きく,
GNSS単体での判定には限界がある.近年では,深層学習を用いて誤差をモデル化するDeepPCO\cite{deepPCO2022}などの手法も提案されているが,
これらは膨大な学習データを必要とする.

また,誤ったデータの影響を抑える手法として,スイッチ変数を用いて不要な
拘束を自動的に無効化するロバスト最適化\cite{6385590}も提案されている.
これは最適化のプロセスにおいて外れ値の影響を最小限に抑えるアプローチである.


\subsection{本研究の特徴}
本研究の特徴は,農業環境で発生するGNSSの大きな誤差を,LIOを基準とした整合性検定に
よって事前に除去する点にある.従来の多くの手法は,最適化計算の内部で誤差に対処するが,
本研究ではデータを投入する前の段階でGNSS拘束を厳しく選別する.
これにより,誤ったデータによるSLAMの破壊を防ぎ,自己位置推定の信頼性を高めている.

\section{農業用生育環境センシングセンサと分光計測}
環境情報を地図上に正しく配置するためには,
計測時の自己位置が正確である必要がある.
自己位置の誤差はマップの品質に直結するため,
信頼性の高い自己位置推定は,高精度な生育環境マップを作成する上で不可欠な要素である.

精密農業において,圃場内の微気象(温度,湿度,CO$_2$濃度など)を把握することは重要である.
固定式のセンサノードは広く普及しているが,設置コストや電源確保の面から,
細かな空間分解能でデータを取得することには限界がある.
これに対し,移動ロボットにセンサを搭載して巡回計測を行うことで,
より高密度な環境マップを作成する試みが進んでいる\cite{ref:mobile_sensing_review}.

また,作物の状態を傷つけずに診断する手法として,
分光反射率の計測が注目されている.近年では,MEMS技術を活用した小型分光センサ(C12880MA等)が登場し,
ドローンやロボットへの搭載が容易になった\cite{ref:drone_spectral}.

先行研究\cite{ref:Kobayashi2021AGV}では,このセンサを用いた計測システムが構築されたが,
移動しながらの連続的な計測が難しいことや,回路設計の制約によりデータ取得速度が \SI{50}{\kilo\hertz} 程度に留まるという課題があった.
しかし,センサ自体は \SI{}{\mega\hertz} オーダの高速動作が可能な設計である.
したがって,これらの制限はセンサ自体の性能ではなく,読み出し回路側の設計に起因するものと考えられる.
\section{本研究の位置づけと新規性}
本研究の新規性は,農業環境におけるGNSS品質の解析,
監視モジュールの提案,およびセンサユニットの統合にある.

まず,ハウス環境のデータからNLOS等の外れ値の発生傾向を整理し,
自己位置推定の破綻を招く条件を明確にした.これに基づき,
受信機の状量だけでなくLIOとの整合性によってGNSS信号を評価する監視モジュールを提案し,
信号劣化時でも地図の破壊を防ぎつつ正確な位置補正を行う手法を実現した.
さらに,分光センサC12880MAの駆動方式を最適化することで \SI{5}{\mega\hertz} の高速サンプリングを達成し,
これらを統合して環境マッピングを実証する.

\section{用語の定義}
本節では,本文中で頻出する専門用語について,
本論文における意味を簡潔に定義する.


  \noindent\textbf{GNSS}:Global Navigation Satellite Systemの総称であり,本論文では,衛星測位に基づく位置観測を一般に指す場合にGNSSと表記する.

  \noindent\textbf{RTK-GNSS}:RTK(Real-Time Kinematic)差分補正を用いるGNSS測位を指す.本研究の受信機はRTK対応であるが,測位解は環境・補正情報の状態によりRTK-FIX / RTK-FLOAT / No-Fixへ遷移し得る.以降,RTK補正を前提とする受信機出力を述べる箇所ではRTK-GNSSと表記し,補正状態を区別しない一般論・総称として述べる箇所ではGNSSと表記する.

  \noindent\textbf{ROS 2}:Robot Operating System 2の略称であり,ロボットソフトウェア開発のためのミドルウェアである.

  \noindent\textbf{topic/トピック}:ROS 2における通信の単位であり,ROS 2ノード間でメッセージを送受信するための論理チャネルである.

  \noindent\textbf{ROS 2 bag(rosbag2)}:ROS 2におけるデータ記録機構であり,topic通信を時系列ログとして保存したデータを指す.本論文では,実験時に取得したセンサデータおよび推定結果の記録に用いる.

  \noindent\textbf{tf}:ROS 2において座標系間の位置・姿勢関係を管理する仕組みである.本研究では,\texttt{map},\texttt{odom},\texttt{base\_link}などの座標系を用いてロボットの状態を表現する.

  \noindent\textbf{\texttt{cmd\_vel}}:ロボットの速度指令を表すROS 2 topicであり,並進速度および角速度を含む.本研究では,自律走行モジュールから車体へ送信される制御入力として用いる.

  \noindent\textbf{SLAM}:Simultaneous Localization and Mappingの略称であり,自己位置推定と環境地図生成を同時に行う枠組みである.

  \noindent\textbf{Foxglove}:ROS 2の可視化ツールであり,実行中のノードやトピックをリアルタイムで監視・可視化する.本研究では,実験データの可視化およびデバッグに用いる.

  \noindent\textbf{NIR}:Near-Infraredの略称であり,分光センサで取得される近赤外線波長帯(およそ700~1400\,nm)を指す.

  \noindent\textbf{RED帯域}:Red波長帯(およそ600~700\,nm)を指す.

  \noindent\textbf{正規化差分(ND)}:正規化差分(Normalized Difference, ND)は,2つのスペクトル成分の相対的な差を表す指標であり,照度変化の影響を低減する目的で用いられる.本研究では,NIR帯域とRed帯域の値を用いて正規化差分を算出し,スペクトル形状の違いを相対的に評価する.


\end{document}