%! TEX root = ../main.tex
\documentclass[main]{subfiles}

\begin{document}

\chapter{生育環境マッピング}
本章では,開発したセンサユニットと位置推定システムを統合し,
実際の圃場環境における生育情報のマッピングを行う.

\section{マッピング手法と実験条件}
本研究の最終目標は完全自律走行によるモニタリングであるが,
本実験の段階では自律走行アルゴリズムの検証を並行して行っているため,
AGVを手動操作して走行させ,データを取得した.
具体的には,GNSS/LiDARに基づく自己位置推定の結果を基準時刻系とし,
非同期に取得される環境センサの観測値を補間・同期することで,
同一の走行データとして統合し,生育情報マップを生成する.
これにより,将来的な自律走行システムにおいて要求される環境マッピング機能の実現可能性を示す.

\section{データの同期と生育環境}
具体的なデータ処理手順は以下の通りである.
\begin{enumerate}
    \item データの同期:
    センサユニットから得られた環境データに対し,
    時刻情報をもとに最も近い自己位置を割り当てる.

    \item 正規化:
    本章のマップでは,場所によるわずかな変化を見やすくするため,
    取得データの範囲に合わせて色の表示レンジを調整する.
    なお,これはあくまで視認性を良くするための処理であり,数値の比較は各センサの出力値そのものに基づく.
    \item 配置と可視化:
    2次元平面上の走行軌跡としてプロットし,
    各地点におけるセンサ値を色情報として重ねて表示する.
\end{enumerate}

\subsection{分光スペクトルの特徴抽出と空間マッピング}
分光センサは各観測時刻$t_i$において288点のスペクトル$\bm{I}_i\in\mathbb{R}^{288}$を出力する.
スペクトルは情報量が多いため,
本研究では前処理,特徴量の抽出,位置情報との対応付け,
という3段階の処理を経てマップ化を行った.

まず,ノイズや明るさの変動を抑えるための補正と正規化を行い,
正規化スペクトル$\bm{s}_i$を得た.
次に,植生とそれ以外を見分けるための指標として,
NIR帯域(近赤外)とRed帯域(赤)の正規化差分(ND),
スペクトル質心$\lambda_c$,スペクトルの傾きが最大となる波長$\lambda_{\mathrm{RE}}$,
および参照データ$\bm{s}_{ref}$に対する類似度 $\theta$(SAM手法)の4つの特徴量を計算した.
最後に,これらの特徴量を取得時刻に最も近い自己位置$\bm{p}(t_i)$と結びつけることで,
走行軌跡上に配置し,空間的な分布として可視化した.

\subsection{スペクトル特徴による植生・非植生の識別可能性}
本研究で用いる分光特徴量は,センサが観測した反射スペクトルに基づく相対的な指標であり,
観測方向や対象面の姿勢,照明条件の影響を受ける.
そのため,気温のような地点固有の状態量を直接表すものではない.
本章では,同一走行条件下で取得した観測量として扱い,空間的な変化傾向を可視化することを目的とする.


288のデータから計算した各特徴量のマップをFig.~\ref{fig:spec_feature_maps}に示す.
NDやスペクトルの傾きが最大となる波長は植物の特徴を強く反映する指標であるため,
マップ上においても植生があるエリアとないエリアの違いがある程度明確に確認できた.
また,SAMやスペクトル質心は,スペクトルの形状変化を表す補助的な指標として用いた.
なお,Fig.~\ref{fig:map_spectral}は参考として作成したピーク波長のマップである.

\subsection{環境データと場所による変化}
次に,環境センサのマッピング結果Fig.~\ref{fig:map_temp}--\ref{fig:map_pressure}に示す.
実験場所は通気性の高いハウスであり,
場所による環境差は非常に小さい条件であったが,マップ化することによっていくつかの特徴が確認できた.

具体的には,Fig.~\ref{fig:map_temp}とFig.~\ref{fig:map_humidity}を比較すると,温度が高い場所では湿度が低く,
温度が低い場所では湿度が高いという逆相関の傾向が見られた.これは,気温の上昇に伴い相対湿度が下がるという物理法則と整合している.
また,Fig.~\ref{fig:map_co2}に示すように,CO$_2$ 濃度はおおよそ400--510\,ppmの範囲で推移している.


一方で,Fig.~\ref{fig:map_pressure} に示す気圧については,
開放型ハウスでの短時間計測であったため,変動幅は約 1~hPa 程度に留まり,
顕著な空間的勾配は見られなかった.


\section{生成された生育環境マップ}
10月23日に収集したデータに基づき生成された各種環境マップを以下に示す.
実験当日の天候は晴れであり,ハウス側面は換気のため開放された状態であった.

% --- Temperature & Humidity ---
\begin{figure}[htbp]
    \centering
    \begin{minipage}[b]{0.48\textwidth}
        \centering
        \includegraphics[width=\textwidth]{figures/5/bag_20251023_112054_agri_xy_temp.pdf}
        \caption{Temperature map ($^\circ$C).}
        \label{fig:map_temp}
    \end{minipage}
    \hfill
    \begin{minipage}[b]{0.48\textwidth}
        \centering
        \includegraphics[width=\textwidth]{figures/5/bag_20251023_112054_agri_xy_humidity.pdf}
        \caption{Relative humidity map (\%).}
        \label{fig:map_humidity}
    \end{minipage}
\end{figure}

% --- CO2 & Pressure ---
\begin{figure}[htbp]
    \centering
    \begin{minipage}[b]{0.48\textwidth}
        \centering
        \includegraphics[width=\textwidth]{figures/5/bag_20251023_112054_agri_xy_co2.pdf}
        \caption{CO$_2$ concentration map (ppm).}
        \label{fig:map_co2}
    \end{minipage}
    \hfill
    \begin{minipage}[b]{0.48\textwidth}
        \centering
        \includegraphics[width=\textwidth]{figures/5/bag_20251023_112054_agri_xy_pressure.pdf}
        \caption{Atmospheric pressure map (hPa).}
        \label{fig:map_pressure}
    \end{minipage}
\end{figure}

% --- Spectral feature maps (2x2) ---
% NOTE: Require \usepackage{subcaption} in the main preamble.
\begin{figure}[htbp]
  \centering

  \begin{subfigure}[b]{0.48\textwidth}
    \centering
    \includegraphics[width=\textwidth]{figures/5/bag_20251023_112054_agri_xy_spec_ndnr.pdf}
    \subcaption{ND(NIR, Red) map.}
    \label{fig:spec_ndnr}
  \end{subfigure}
  \hfill
  \begin{subfigure}[b]{0.48\textwidth}
    \centering
    \includegraphics[width=\textwidth]{figures/5/bag_20251023_112054_agri_xy_spec_centroid.pdf}
    \subcaption{Spectral centroid map.}
    \label{fig:spec_centroid}
  \end{subfigure}

  \vspace{0.6em}

  \begin{subfigure}[b]{0.48\textwidth}
    \centering
    \includegraphics[width=\textwidth]{figures/5/bag_20251023_112054_agri_xy_spec_rededge.pdf}
    \subcaption{Red-edge position map.}
    \label{fig:spec_rededge}
  \end{subfigure}
  \hfill
  \begin{subfigure}[b]{0.48\textwidth}
    \centering
    \includegraphics[width=\textwidth]{figures/5/bag_20251023_112054_agri_xy_spec_sam_leaf.pdf}
    \subcaption{SAM similarity to vegetation reference.}
    \label{fig:spec_sam}
  \end{subfigure}

  \caption{Spatial maps of spectral features along the trajectory, extracted from 288-band measurements.
            (a) ND(NIR, Red) map.
            (b) Spectral centroid map.
            (c) Red-edge position map.
            (d) SAM similarity map.
            These maps visualize relative variations of observations acquired along the trajectory.}
  \label{fig:spec_feature_maps}
\end{figure}

% --- Spectral peak wavelength (exploratory) ---
\begin{figure}[htbp]
    \centering
    \includegraphics[width=0.75\textwidth]{figures/5/bag_20251023_112054_agri_xy_peaklambda.pdf}
    \caption{Spectral peak wavelength map extracted from 288-band measurements.
The map is shown as an exploratory visualization along the trajectory.}
    \label{fig:map_spectral}
\end{figure}


\section{考察}
\label{sec:discussion}
本実験により,走行中に取得した環境センサデータを自己位置推定結果と同期させ,
圃場内の空間分布として可視化できることを確認した.
以下では,マッピングの幾何学的な正しさと,環境データの妥当性について考察する.

\subsection{自己位置精度のマッピングへの影響}
生成された環境マップの幾何学的な正しさについて述べる.
本実験環境には真値が存在しないため,絶対精度の定量評価は困難である.
しかし,第3章のFig.~\ref{fig:map_comparison_4panel}で示したように,
外れ値による地図の破壊が防げている.
したがって,地図の不連続な破壊が生じていないこと,および走行軌跡が連続に得られていることから,
LIOが少なくとも大きな破綻のない相対精度を維持していると判断できる.

ここで,位置誤差が生育環境マップに与える影響を考える.
ハウス内の温度,湿度,CO$_2$濃度といった環境データは,数cm単位で激変するものではなく,
通常は数mの範囲で緩やかに変化する.
提案手法によりメートル級の突発的な位置飛びは排除されている.
そのため,残留している数十cm程度の累積誤差は,
環境データの変化の規模に比べれば十分に小さいと判断する.

また,分光データ(Fig.~\ref{fig:spec_feature_maps})では,
正規化差分などの特徴量を用いることで,植生の有無を反映したマップが得られた.
特に,NDが高い値を示した区間は,走行経路上の植生が存在する領域に対応していると考えられる.
このことから,提案システムは走行中に取得した観測データを空間分布として整理し,
圃場内の相対的な特徴の違いを可視化できているといえる.

\end{document}
