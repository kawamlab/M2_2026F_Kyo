%! TEX root = ../main.tex
\documentclass[main]{subfiles}

\begin{document}

\chapter{高速分光センシングシステムの構築}
\label{ch:sensor_unit}
本章では,AGV走行中に生育環境を計測するために構築したセンサユニットについて述べる.
ユニットは環境センサ群と分光センサから構成される.
以降は,主要要素である分光センサの高速取得の実装と評価を示す.
実験環境(信州大学繊維学部の圃場)では,
ロボットが走行しながら連続的に計測を行うため,
分光データの取得には安定性が求められる.

本研究で使用する浜松ホトニクス製の超小型分光センサC12880MAは,外部クロックに同期して信号を出力する仕様である.
しかし,一般的な割り込み処理に基づくデータ取得方法では,動作が高速になると処理が追いつかず,
データの欠落やタイミングのズレが生じやすい.
そこで本研究では,まず再現実験によって割り込み処理の限界を示した上で,
DMA(Direct Memory Access,以下,DMA)を用いたデータ取得システムを設計・実装した.

具体的には,タイマを用いてセンサへのクロック(CLK信号)を連続的に生成し,
トリガ信号(TRG信号)に同期してADC変換を開始させる.
変換されたデータは,DMAによってCPUを介さずにメモリへ直接転送される.
さらに,信号(ST)の立ち下がりから計測開始までの遅延時間を実測し,
有効な画素範囲の補正に反映させることで,\SI{5}{\mega\hertz}での安定した連続取得を実現した.
この手法は,マイクロコントローラ(STM32F446REおよびSTM32H723ZG)上に実装されている.

本研究で使用した分光センサの外観をFig.~\ref{fig:c12880ma_photo}に,
システム設計において主な仕様をTable~\ref{tab:c12880ma_spec}に示す.

\begin{figure}[b]
  \centering
  \includegraphics[width=0.5\linewidth]{figures/4/c12880ma.png}
  \caption{Miniature spectrometer C12880MA used in the proposed spectral sensing system}
  \label{fig:c12880ma_photo}
\end{figure}

\begin{table}[t]
  \centering
  \caption{Key specifications of the C12880MA spectrometer relevant to this study}
  \label{tab:c12880ma_spec}
  \begin{tabular}{l c}
    \hline
    Item & Specification \\
    \hline
    Spectral range & 340--850 nm \\
    Number of pixels & 288 \\
    Pixel size & $14 \times 200~\mu$m \\
    Slit size & $50 \times 500~\mu$m \\
    Maximum clock frequency & 5 MHz \\
    Output signal & Analog video \\
    Supply voltage & 5 V \\
    Dimensions & $20.1 \times 12.5 \times 10.1$ mm \\
    Weight & 5 g \\
    \hline
  \end{tabular}
\end{table}

\section{センサユニットの概要}
\label{sec:sensor_unit_overview}
本研究で用いるセンサユニットは,植生状態観察のための分光計測を中核に,
生育環境のデータ(温度・湿度・気圧,\(\mathrm{CO_2}\)濃度)も同時に取得できるセンサユニットを構成した.
センサユニット全体の構成をFig.~\ref{fig:agv_multisensor_overall}に示す.
環境量センサの実装および計測ロジックは先行研究で確立されているため,
本章ではC12880MAの高速駆動を実現するための要素技術に焦点を当てる.
各センサの仕様およびMCUとのインタフェース回路については,付録
(Fig.~\ref{fig:mcu_circuit},Fig.~\ref{fig:c12880ma_circuit},
Fig.~\ref{fig:s300l_circuit},Fig.~\ref{fig:bme280_circuit},)に示す.

\begin{figure}[t]
  \centering
  \includegraphics[width=\linewidth]{figures/4/AGV_Multisensor.pdf}
  \caption{Overall multisensor system architecture}
  \label{fig:agv_multisensor_overall}
\end{figure}

センサユニットを構成する部品とその主な仕様をTable~\ref{tab:sensor_unit_specs}に,
開発したユニットの外観をFig.~\ref{fig:sensor_unit_photo}に示す.
C12880MAは外部クロックに同期して信号を出力するため,MCU側でタイミング制御を行う必要がある.
一方,温度・湿度・気圧センサBME280および\(\mathrm{CO_2}\)センサS300L-3Vは
先行研究で確立されたI\(^2\)Cインタフェースを用いて接続されている.
これらのセンサの詳細については,それぞれ\ref{appendix:BME280}および\ref{appendix:S300L}に示す.

\begin{figure}[t]
  \centering
  \includegraphics[width=\linewidth]{figures/4/placeholder_sensor_unit_photo.pdf}
  \caption{Photographs of the developed sensor unit. 
  (A) Top view of the sensor board integrating the C12880MA, BME280, and S300L-3V. 
  (B) Oblique view showing the component layout. 
  (C) Top view with the optical bandpass filter holder mounted. 
  (D) The fully assembled unit ready for installation.}
  \label{fig:sensor_unit_photo}
\end{figure}


\begin{table}[t]
  \centering
  \caption{Sensor unit components and key specifications (overview)}
  \label{tab:sensor_unit_specs}
  \setlength{\tabcolsep}{4pt}
  \renewcommand{\arraystretch}{1.15}
  \begin{tabular}{p{0.32\linewidth} p{0.22\linewidth} p{0.18\linewidth} p{0.22\linewidth}}
    \hline
    Component & Manufacturer & Interface & Note \\
    \hline \hline
    Mini-spectrometer C12880MA
      & Hamamatsu Photonics
      & Analog (VIDEO), CLK/ST/TRG
      & High-speed synchronized acquisition  \\
    BME280 (AE-BME280)
      & Bosch Sensortec / Akizuki Denshi
      & I\(^2\)C
      & Temperature / humidity / pressure \\
    S300L-3V \(\mathrm{CO_2}\) sensor
      & ELT SENSOR
      & I\(^2\)C
      & \(\mathrm{CO_2}\) concentration \\
    MCU
      & STMicroelectronics
      & ---
      & STM32F446RE (F4) / STM32H723ZG (H7) \\
    \hline
  \end{tabular}
\end{table}

\section{従来方式の課題}
C12880MAセンサは,入力信号(ST信号)の立ち下がり後にTRG信号を出力し,
これに同期してビデオ信号(VIDEO信号)を更新する仕様である(Fig.~\ref{fig:c12880ma_timing}).
従来の手法では,このTRG信号の立ち上がりを検知して割込みを発生させ,
その処理ルーチン(ISR)内でADCを起動してデータを読み出していた\cite{ref:Kobayashi2021AGV}.
しかし,5\si{\mega\hertz}という高速なクロックで動作させる場合,
TRG信号の周期は\SI{200}{\nano\second}程度と極めて短くなる.
この時間スケールでは,割込み処理に入るまでの遅延や処理時間のわずかなばらつきが無視できず,
データの取りこぼしやタイミングのズレが頻発する.
したがって,センサ本来の性能を引き出すには,
ソフトウェアによる割込み処理に依存しない,ハードウェアレベルでの同期機構が不可欠である.

本研究では,従来手法と提案手法におけるデータ処理の流れを比較し,
性能の制約を明確にする(Fig.~\ref{fig:acq_arch_compare}).
従来方式では,TRG信号ごとにCPUが介入する必要があるため,高速動作時にはCPUの処理能力が追いつかなくなる.
これに対し提案方式では,タイマとDMAを連携させることで,CPUを介さずにADC変換とデータ転送を自動的に行う仕組みを採用した.


\begin{figure}[t]
  \centering
  \begin{subfigure}[b]{0.49\linewidth}
    \centering
    \includegraphics[width=\linewidth]{figures/4/isr.png}
    \caption{Interrupt-driven acquisition (conventional)}
    \label{fig:acq_isr_conventional}
  \end{subfigure}
  \hfill
  \begin{subfigure}[b]{0.49\linewidth}
    \centering
    \includegraphics[width=\linewidth]{figures/4/input_capture.png}
    \caption{Timer-triggered acquisition with DMA (proposed)}
    \label{fig:acq_hwtrig_proposed}
  \end{subfigure}

  \caption{Comparison of acquisition data paths.
  In the conventional method, the CPU handles every TRG event in an ISR, which becomes the bottleneck at MHz rates.
  In the proposed method, the TRG edge is captured by a timer and directly drives ADC conversion via an external trigger, while DMA streams samples to memory; CPU involvement is limited to frame start/end control.}
  \label{fig:acq_arch_compare}
\end{figure}

以上の比較から,高周波領域における主要な性能制約が周期ごとのCPU介在に起因することが分かる.
次節では,STM32F446REを用いて従来方式の限界を再現的に確認し(Table~\ref{tab:previous_method_limit}),
その後,提案方式により制約を回避できることを示す.



\begin{figure}[ht]
  \centering
  \includegraphics[width=0.9\linewidth]{figures/4/placeholder_C12880MA_timing.png}
  \caption{Timing diagram of the C12880MA: excerpted from the datasheet\cite{ref:C12880MA}}
  \label{fig:c12880ma_timing}
\end{figure}

\begin{figure}[t]
  \centering
  \includegraphics[width=0.98\linewidth]{figures/4/c12880ma_pixel_index_mapping.pdf}
  \caption{C12880MA readout rule and pixel index mapping between sensor-side definition and DMA buffer in this work. 
  The sensor pixel index is defined from the falling edge of ST (\(t=0\)), where pixels \#1--\#88 are invalid and pixels \#89--\#376 are valid (288 pixels). 
  Due to software start latency, sensor pixels \#1--\#4 are not captured. therefore, the nominal valid start at sensor pixel \#89 appears at DMA buffer sample \#85.}
  \label{fig:c12880ma_readout_rule}
\end{figure}


Fig.~\ref{fig:c12880ma_readout_rule}に,本研究で用いる画素番号の定義と,
センサ側インデックスとDMAバッファ上のサンプル番号の対応関係を示す.
本研究ではST信号の立下りを\(t=0\)とし,その直後の第1画素を\#1と定義する.
データシート上の仕様では,\#1--\#88は無効画素であり,\#89--\#376 が有効画素である.

一方,実装では,ST信号を立ち下げた後にソフトウェアでADCおよびDMAを起動するため,処理遅延が発生する.
実測の結果,この起動遅延によりセンサ側の \#1--\#4 に相当するデータが取得できていないことが判明した.
その結果,本来\#89 から始まる有効データは,DMAバッファ上では\#85 の位置に出現する.
本研究では,この\#85--\#372(計288サンプル)を有効な分光データとして採用する.
なお,EOS(End of Scan)信号は読み出し終了の確認にのみ使用し,
DMA転送自体は固定長(\(N=387\),無効画素を含む1フレーム分で終了させる設計とした.

また,従来方式の限界を確認するため,内部ADCが比較的高速なSTM32F446REを
用いて検証実験を行った(Table~\ref{tab:previous_method_limit}).
割込み処理のみを用いた場合,\SI{25.4}{\kilo\hertz}付近でデータの欠落が発生した.
DMAを併用してCPU負荷を低減した場合でも,\SI{130}{\kilo\hertz}程度が限界であった.
この比較実験により,CPUによるソフトウェア処理が,高速化を妨げる主な原因であることを確認した.

\begin{table}[t]
    \centering
    \caption{Limits of acquisition frequency with conventional methods}
    \label{tab:previous_method_limit}
    \begin{tabular}{l|l|c}
        \hline
        \textbf{Method} & \textbf{MCU} & \textbf{Achieved frequency} \\
        \hline \hline
        Interrupt only      & STM32F446RE & \SI{25.4}{\kilo\hertz}\\
        Interrupt + DMA     & STM32F446RE & \SI{130}{\kilo\hertz} \\
        \hline
    \end{tabular}
\end{table}

\section{分光データのハードウェア同期取得アーキテクチャ}
\label{sec:proposed_chain}
本研究では,\SI{5}{\mega\hertz}級の高速読み出しにおいてサンプリング位相の決定性を確保するため,
ソフトウェア割込みを介在させないハードウェア同期取得を構築した.
C12880MAは,CLK信号を連続的に入力し,ST信号によって積分区間を制御する仕様である.
提案手法では,TRG信号に同期したハードウェアトリガでADC変換を直接駆動し,
DMAを用いて画素データをス逐次転送する.

\subsection{取得タイミング設計}
\label{subsec:acq_sequence}
1フレームの取得は次の手順で構成される.
(i) STをHighに設定して積分を開始し,所定の積分時間 \(T_{\mathrm{int}}\) だけ待機する.
(ii) STをLowへ遷移させて積分を終了し,直後にADCをDMAモードで開始する.
(iii) 以降はTRG信号の立上りに同期してADC変換が自動的に実行され,DMAによりバッファへ連続転送される.
(iv) DMA転送完了MCUのcallbackによりフレーム完了を検知し,次フレーム準備としてSTをHighへ戻す.
この構成により,
各サンプルの取得タイミングはTRG信号によって規定され,
CPUの介入はフレームの開始と終了のみに限定される.

\subsection{サンプリング実装}
\label{subsec:tim_adc_dma_link}
Fig.~\ref{fig:c12880ma_timing}に示した通り,
C12880MAは入力されるCLK信号に同期してTRG信号を出力する.
本実装では,汎用タイマ(TIM2)のPWM出力機能を用いてセンサへのCLK信号を生成する.
さらに,別のタイマ(TIM15)を用いてTRG信号のエッジを検出し,そのイベントをADCの外部トリガとして接続した.
これにより,TRG信号の周期に同期した正確なサンプリングが,CPUの処理状況に依存せず実行される.

\subsection{有効画素範囲の補正}
\label{subsec:effective_pixel_window}
前述の通り,本実装ではST信号の立ち下がり後にソフトウェアでADCとDMAを開始するため,わずかな起動遅延が生じる.
この遅延は,取得される画素列の開始位置における固定のオフセットとして現れる.
実測の結果,このオフセットはTRG周期換算で約4画素分(\#1--\#4)であることが確認された.
したがって,DMAバッファ上では本来89番目にあるはずの有効画素の先頭が,85番目付近に出現する.
本研究ではこの挙動を固定遅延として扱い,
バッファの先頭85画素分を無効データとして除外することで,有効なスペクトルデータを安定して抽出した.

% \subsection{論理ゲートによる完全同期への拡張}
% \label{subsec:and_gate_future}
% 本章の評価では,高速取得中に追加の高負荷割り込み処理(通信スタックや制御周期割り込み等)を介在させず,
% 時間決定性を優先した構成としている.この前提下では,前節の画素オフセットはほぼ一定であり,
% 有効画素ウィンドウ補正として安定に吸収できる.

% 一方,将来的に割り込み負荷が増大し,ST遷移からADC/DMA開始までのレイテンシが変動する場合,
% 画素シフトが非定常となり,定数補正のみでは不十分となる可能性がある.
% その場合には,ST信号と「ADC/DMAのARM完了(開始準備完了)」状態を論理回路(例:与ゲート)で合成し,
% 「ST=LowかつARM完了」の条件成立後にのみTRG(もしくはADCトリガ経路)を有効化する
% ハードウェアゲーティングにより,フレーム開始位置を完全に決定論的に固定できる.
% 本研究は現段階でこの拡張を要さないが,システム統合要件の変化に応じて導入可能である.

\section{評価}
\label{sec:performance_eval}

本節では,提案アーキテクチャが安定したデータ取得を実現していることを検証する.
具体的には,まずオシロスコープを用いた信号タイミングの実測を行い,ハードウェアレベルでの同期動作を確認する.
次に,連続取得時におけるデータの完全性,すなわちデータの欠落やズレがないかを検証する.
最後に,既知の光源を用いたスペクトルの再現性を評価する.
なお,本実験で使用したSTM32 NUCLEO-H723ZG開発ボードとセンサユニットの接続構成をFig.~\ref{fig:eval_setup_h723} に示す.

\subsection{高速動作の検証}
\label{subsec:eval_h7}
本研究では最大 \SI{5}{\mega\hertz} のサンプリング速度を実現するため,STM32H723ZGに提案アーキテクチャを実装した.
STM32H7シリーズではデータキャッシュが有効に機能するため,
DMAがメモリに書き込んだ最新データをCPUが即座に読み出せないキャッシュ・コヒーレンシの問題が発生する場合がある.
そこで本実装では,DMA の転送先バッファをキャッシュの影響を受けないDTCM(Data Tightly Coupled Memory)領域に配置することで,
データの整合性を確保した.
加えて,ADCのハードウェアキャリブレーションとトリガ設定を最適化することで,目標とする\SI{5}{\mega\hertz}での連続取得に成功した.

Fig.~\ref{fig:scope_clk_trg_video_eos}に,\SI{5}{\mega\hertz}動作時におけるオシロスコープの計測画面を示す.
ここではCH1をTRG信号,CH2をCLK信号,CH3をEOS信号,CH4をVIDEO信号としている.
波形を確認すると,TRG の立ち上がりに同期して VIDEO信号が遷移している様子や,
EOS信号の発生付近でデータ出力が停止する正常な挙動が確認できる.

\begin{figure}[t]
  \centering
  \begin{subfigure}[b]{0.49\linewidth}
    \centering
    \includegraphics[width=\linewidth]{figures/4/SCRN0038.PNG}
    \caption{Strong illumination}
    \label{fig:eos_strong}
  \end{subfigure}
  \hfill
  \begin{subfigure}[b]{0.49\linewidth}
    \centering
    \includegraphics[width=\linewidth]{figures/4/SCRN0037.PNG}
    \caption{Dark condition}
    \label{fig:eos_dark}
  \end{subfigure}
  
  \caption{Oscilloscope screenshots of the End-of-Scan (EOS) timing at 5~MHz.
(a) Strong illumination.
(b) Dark condition.
Yellow: TRG, magenta: CLK.}
  \label{fig:scope_clk_trg_video_eos}
\end{figure}

\subsection{取得データの妥当性確認}
本節では,高速分光センシングシステムにおいて取得される信号が,
意図したタイミング制御および読み出し制御に従って動作しているかを確認することを目的とする.
オシロスコープから保存した CSV データを用い,
EOS 信号,ST 信号および VIDEO 信号の関係をグラフとしてプロットした.

まず,Fig.~\ref{fig:eos_video_high_low} に EOS 信号と VIDEO 信号の波形を示す.
本実験では,光源とセンサ間の配置を一定とし,照明条件のみを変化させた.
強い光を当てた場合(Fig.~\ref{fig:eos_video_high_low}(a))と
遮光した場合(Fig.~\ref{fig:eos_video_high_low}(b))のいずれにおいても,
EOS信号の立ち上がりに同期してVIDEO出力が停止している.
これにより,ゲートロジックによる読み出し停止機能が
外部環境光条件によらず正しく動作していることを確認した.

次に,様々な照明環境および光源距離における
ST信号とVIDEO 信号の応答特性を
Fig.~\ref{fig:st_video_combined_conditions}に示す.
Fig.~\ref{fig:st_video_combined_conditions}(a)の遮光条件では
ノイズ成分のみが観測されているのに対し,
同(b)の自然光下では光源に対応したスペクトルピークが捉えられている.
また,光源との距離を変えた同(c)および同(d)を比較すると,
光量に応じてVIDEO 信号の振幅が変化しており,
本システムが光量変化に対する基本的な応答性を有していることが分かる.


\begin{figure}[h]
  \centering
  \begin{subfigure}[b]{0.49\linewidth}
    \centering
    \includegraphics[width=\linewidth]{figures/4/eos_video_high.png}
    \caption{Strong illumination}
    \label{fig:eos_high}
  \end{subfigure}
  \hfill
  \begin{subfigure}[b]{0.49\linewidth}
    \centering
    \includegraphics[width=\linewidth]{figures/4/eos_video_low.png}
    \caption{Dark condition}
    \label{fig:eos_low}
  \end{subfigure}
  \caption{Replotted EOS (green) and VIDEO (red) waveforms from CSV data at 5~MHz.
          (a) Strong illumination.
          (b) Dark condition.}
  \label{fig:eos_video_high_low}
\end{figure}

\begin{figure}[t]
  \centering
  % Row 1: Baseline and Natural Light
  \begin{subfigure}[b]{0.48\linewidth}
    \centering
    \includegraphics[width=\linewidth]{figures/4/st_video_black.png}
    \caption{Dark condition}
    \label{fig:cond_dark}
  \end{subfigure}
  \hfill
  \begin{subfigure}[b]{0.48\linewidth}
    \centering
    \includegraphics[width=\linewidth]{figures/4/st_video_nature_light.png}
    \caption{Natural light}
    \label{fig:cond_nature}
  \end{subfigure}
  
  \vspace{1em} % Add some vertical space between rows

  % Row 2: Distance/Intensity comparison
  \begin{subfigure}[b]{0.48\linewidth}
    \centering
    \includegraphics[width=\linewidth]{figures/4/st_video_5cm_light.png}
    \caption{Strong illumination (\SI{5}{cm})}
    \label{fig:cond_5cm}
  \end{subfigure}
  \hfill
  \begin{subfigure}[b]{0.48\linewidth}
    \centering
    \includegraphics[width=\linewidth]{figures/4/st_video_25cm_light.png}
    \caption{Low illumination (\SI{25}{cm})}
    \label{fig:cond_25cm}
  \end{subfigure}
  
  \caption{ST and VIDEO signal waveforms under different illumination conditions.
          (a) Dark condition.
          (b) Natural light.
          (c) Strong illumination at 5~cm.
          (d) Low illumination at 25~cm.} 
  \label{fig:st_video_combined_conditions}
\end{figure}

% \subsection{サンプリングレート成立性の検討}
% \label{subsec:adc_feasibility}
% センサに供給するCLK信号が\SI{5}{\mega\hertz}であるため,1画素あたりの更新周期は
% \[
% T_{\mathrm{period}}=\frac{1}{\SI{5}{\mega\hertz}}=\SI{200}{\nano\second}
% \]
% となる.データシートによれば,VIDEO 信号が安定しているのはTRG信号の立ち上がりを中心とした半周期程度であるため,
% ADCによるサンプリングが可能な時間幅は概ね\(\SI{100}{\nano\second}\)に制限される.
% 本実装ではサンプリング時間に\(\SI{2.5}{}\)サイクル,変換時間に\(\SI{12.5}{}\)サイクル(12-bit分解能)を設定しており,合計で15サイクルを要する.
% したがって,必要な ADC クロック周波数 \(f_{\mathrm{ADCK}}\) は以下のように求められる.
% \begin{equation}
%   f_{\mathrm{ADCK}} > \frac{15}{\SI{200}{\nano\second}} = \SI{75}{\mega\hertz}
% \end{equation}
% STM32H723ZGのデータシートにおいて,12-bit ADCの最大クロック周波数は \SI{75}{\mega\hertz} と規定されている.
% これにより,本研究の設定がハードウェアの仕様範囲内で成立していることを確認した.
% 実測においても,この条件下で安定してデータが取得できている

\begin{table}[!tbp]
  \centering
  \caption{Acquisition frequency of the proposed architecture}
  \label{tab:acq_freq_proposed}
  \setlength{\tabcolsep}{5pt}
  \renewcommand{\arraystretch}{1.15}
  \begin{tabular}{l l l}
    \toprule
    Method & MCU & Achieved frequency \\
    \midrule
    Proposed architecture & STM32F446RE & \SI{1.5}{\mega\hertz} (theory) \\
    Proposed architecture & STM32H723ZG & \SI{5.0}{\mega\hertz} (achieved) \\
    \bottomrule
  \end{tabular}
\end{table}


\section{その他センサのデータ取得と統合}
\label{sec:other_sensors}
分光データと同時に,生育環境の基礎量として温度・湿度・気圧および\(\mathrm{CO_2}\)濃度を取得するため,
BME280およびS300L-3Vをシステムに統合した.これら環境センサの取得処理そのものは先行研究で確立された実装に基づくが,
本研究ではセンサユニットのMCUをSTM32へ変更したことに伴い,
デバイスドライバおよび制御を再設計し,
既存処理をSTM32環境へポーティングした.具体的には,S300L-3VやBME280はI\(^2\)C通信により所定周期で計測する.
各計測値にはMCU内の単調増加タイムスタンプを付与し,
分光フレームと同一の時刻系でログ化することで,走行中の計測データとして統合可能とした.
% 収集したデータは上位計算機へ周期的に送信し記録する.通信方式およびデータフォーマットの詳細は,
% 再現性確保の観点から付録にまとめる.

\section{まとめ}
\label{sec:sensor_unit_summary}
本章では,C12880MAを用いた高速分光センシングの実現に向け,
ハードウェア同期に基づくデータ取得システムを構築した.
まず,従来の割込み駆動方式では高速化に限界があることを実験により示した.
その上で,タイマとDMAを連携させ,CPU を介さずに連続的に画素データを取得するアーキテクチャを提案した.
また,読み出し開始時の遅延によって生じる画素ズレを定量化し,有効画素範囲を補正することで,正確なスペクトル計測を可能にした.
最後STM32H723ZGに実装し,\SI{5}{\mega\hertz}での安定した連続取得を実証した.

次章では,本センサユニットを用いて取得したデータに基づき,環境マッピングおよびその評価を行った結果について述べる.

\end{document}